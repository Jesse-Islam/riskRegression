% Created 2019-06-24 Mon 17:59
% Intended LaTeX compiler: pdflatex
\documentclass{article}
               \usepackage{listings}
\usepackage{color}
\usepackage{amsmath}
\usepackage{array}
\usepackage[T1]{fontenc}
\usepackage{natbib}
               \usepackage{authblk}
\usepackage{natbib}
\author{Anders Munch and Thomas Alexander Gerds}
\affil{University of Copenhagen, Department of Public Health, Section of Biostatistics, Copenhagen, Denmark}

\lstset{
keywordstyle=\color{blue},
commentstyle=\color{red},stringstyle=\color[rgb]{0,.5,0},
literate={~}{$\sim$}{1},
basicstyle=\ttfamily\small,
columns=fullflexible,
breaklines=true,
breakatwhitespace=false,
numbers=left,
numberstyle=\ttfamily\tiny\color{gray},
stepnumber=1,
numbersep=10pt,
backgroundcolor=\color{white},
tabsize=4,
keepspaces=true,
showspaces=false,
showstringspaces=false,
xleftmargin=.23in,
frame=single,
basewidth={0.5em,0.4em},
}

\usepackage[utf8]{inputenc}
\usepackage[T1]{fontenc}
\usepackage{graphicx}
\usepackage{grffile}
\usepackage{longtable}
\usepackage{wrapfig}
\usepackage{rotating}
\usepackage[normalem]{ulem}
\usepackage{amsmath}
\usepackage{textcomp}
\usepackage{amssymb}
\usepackage{capt-of}
\usepackage{hyperref}
\date{\today}
\title{k-fold cross-validation}
\begin{document}

\maketitle

\section{Introduction}
\label{sec:orgce8d327}

This document discusses the different cross-validation options of the
\texttt{Score} function.

\section{Formula}
\label{sec:orgdd3b9ce}

The observed outcome status at time \(t\) for subject \(i\) is
\(\tilde Y_i=1_{\{\min(T_i,C_i)\le t\}}\) where \(T_i\) is the event
time and \(C_i\) the right censoring time. A risk prediction is a
value between 0 and 1. The risk prediction of a statistical model at
time \(t\) for subject \(i\) is based on baseline predictor variables
\(X_i\) and given by \(\hat R(t|X_i)\). The inverse probability of
censoring weights (IPCW) are given by \(\hat W_i(t)\) based on a model for
the conditional censoring survival distribution \(P(C> t|X)\).

The data \(D_n=\{(\tilde T_i,\Delta_i,X_i)\}_{i=1}^n\), where also
\(\tilde T_i=\min (T_i,C_i)\) and \(\Delta_i=1_{\{T_i\le C_i\}})\) are
divided into disjoint sets:
\begin{equation*}
D_n = \underbrace{D_l}_{\text{Training set}} \cup \underbrace{D_v}_{\text{Validation set}}
\end{equation*}

\subsection{\texttt{loob}}
\label{sec:orga7ecde7}

\begin{itemize}
\item \texttt{B} number of bootstrap samples
\item \(D_l^b\) = b'th bootstrap sample
\item \(D_v^b=D_n \setminus D_l^b\)  = samples out-of-bag
\item \(\hat R_b\) the model fitted in \(D_l^b\)
\item \(0\le r_i \le B\) the number of bootstrap samples where subject \(i\) is out-of-bag
\end{itemize}
\begin{equation*}
\text{loob}=\frac 1 n \sum_{i=1}^n\frac{1}{r_i}
\sum_{b: i\in D_n\setminus D^b_l}\hat W_i(t) \{\tilde Y_i(t)-\hat
R_b(t|X_i)\}^2
\end{equation*}

\subsection{\texttt{bootcv}}
\label{sec:org688d21f}

\begin{itemize}
\item \texttt{B} number of bootstrap samples
\item \(D_l^b\) = b'th bootstrap sample
\item \(D_v^b=D_n \setminus D_l^b\)  = samples out-of-bag
\item \(\hat R_b\) the model fitted in \(D_l^b\)
\item \(0\le m_b \le n\) the size of \(D_v^b\)
\end{itemize}

\begin{equation*}
  \text{bootcv}= \frac 1 B \sum_{b=1}^B\frac{1}{m_b}
  \sum_{i\in D_n\setminus D^b_l}\hat
  W_i(t) \{\tilde Y_i(t)-\hat R_b(t|X_i)\}^2.
\end{equation*}

\subsection{\texttt{cv-K} once}
\label{sec:org03587f7}

\begin{itemize}
\item \texttt{K} number of folds
\item \(D_l^k=D_n \setminus D_v^k\) = data without fold-k
\item \(D_v^k\)  = data in fold-k
\item \(\hat R_k\) the model fitted in \(D_l^k\)
\item \(\hat R_{k_i}\) the prediction of model fitted without the fold \(k_i\) where \(i\in D_v^k\)
\end{itemize}

\begin{equation*}
\text{cv-K}=  \frac{1}{n} \sum_{i\in D_n}
  \hat W_i(t) \{\tilde Y_i(t)-\hat R_{k_i}(t|X_i)\}^2.
\end{equation*}

\subsection{\texttt{cv-K} repeated B times}
\label{sec:org790fb94}

Same as \texttt{cv-K} but now we repeat K-fold B times. At each time we use a
different seed to select the \(k\) folds of the data. We use the notation:

\begin{itemize}
\item \(D_v^{b,k}\) is the data in the fold-\(k\) at the \(b\)'th iteration.
\item \(D_l^{b, k} = D^n \setminus D_v^{b,k}\)  is the data
without the fold-\(k\) selected at iteration \(b\).
\item \(\hat{R}^b_k\) is a model trained on \(D_l^{b, k}\), i.e., on whole
data except fold-\(k\) from the \(b\)'th iteration.
\item For each \(i = 1, \dots, n\) and \(b = 1, \dots, B\), let \(k_i^b\) be
defined by \(i \in D^{b, k_i^b}_v\), i.e., \(k_i^b\) denotes the fold in
the \(b\) iteration that subject \(i\) is part of. Then \(\hat{R}^b_{k_i} :=
  \hat{R}^b_{k^b_i}\) denotes the model trained on the all the folds of the
\(b\)'th iteration, except the fold including subject \(i\).
\end{itemize}

Then, we have two
possibilities:

\begin{enumerate}
\item Average the B values of \texttt{cv-K} (like \texttt{bootcv}).
\item For each \(i\) collect the B predictions (like \texttt{loob}).
\end{enumerate}

Because we use the same data set in every iteration, each subject \(i\)
is present in a validation set exactly \(B\) times. Therefore these two
approaches give the same formula:
\begin{equation*}
  \frac{1}{B}\sum_{b=1}^{B}\frac{1}{n}\sum_{i \in D_n} \hat{W}_i(t)
  \left ( \tilde{Y}_i(t) - \hat{R}_{k_i}^b(t \mid X_i)  \right)^2 .
\end{equation*}


\section{Testing}
\label{sec:orgee1284b}
Copy/paste some functionality form other vignette.

Packages and source for reloading after edits.
\lstset{language=r,label= ,caption= ,captionpos=b,numbers=none}
\begin{lstlisting}
library(riskRegression)
sessionInfo()
\end{lstlisting}

\begin{verbatim}
Registered S3 methods overwritten by 'ggplot2':
  method         from
  [.quosures     rlang
  c.quosures     rlang
  print.quosures rlang
riskRegression version 2019.06.13
R version 3.6.0 (2019-04-26)
Platform: x86_64-apple-darwin15.6.0 (64-bit)
Running under: OS X El Capitan 10.11.6

Matrix products: default
BLAS:   /Library/Frameworks/R.framework/Versions/3.6/Resources/lib/libRblas.0.dylib
LAPACK: /Library/Frameworks/R.framework/Versions/3.6/Resources/lib/libRlapack.dylib

locale:
[1] en_US.UTF-8/en_US.UTF-8/en_US.UTF-8/C/en_US.UTF-8/en_US.UTF-8

attached base packages:
[1] stats     graphics  grDevices utils     datasets  methods   base

other attached packages:
[1] riskRegression_2019.06.13

loaded via a namespace (and not attached):
 [1] Rcpp_1.0.1          mvtnorm_1.0-10      lattice_0.20-38
 [4] zoo_1.8-6           assertthat_0.2.1    digest_0.6.19
 [7] foreach_1.4.4       R6_2.4.0            plyr_1.8.4
[10] backports_1.1.4     acepack_1.4.1       MatrixModels_0.4-1
[13] ggplot2_3.1.1       pillar_1.4.1        rlang_0.3.4
[16] lazyeval_0.2.2      multcomp_1.4-10     rstudioapi_0.10
[19] data.table_1.12.2   SparseM_1.77        rpart_4.1-15
[22] Matrix_1.2-17       checkmate_1.9.3     splines_3.6.0
[25] stringr_1.4.0       foreign_0.8-71      htmlwidgets_1.3
[28] munsell_0.5.0       numDeriv_2016.8-1.1 compiler_3.6.0
[31] xfun_0.7            pkgconfig_2.0.2     base64enc_0.1-3
[34] htmltools_0.3.6     nnet_7.3-12         tidyselect_0.2.5
[37] tibble_2.1.3        gridExtra_2.3       htmlTable_1.13.1
[40] prodlim_2018.04.18  Hmisc_4.2-0         rms_5.1-3.1
[43] codetools_0.2-16    crayon_1.3.4        dplyr_0.8.1
[46] MASS_7.3-51.4       timereg_1.9.3       grid_3.6.0
[49] nlme_3.1-140        polspline_1.1.14    gtable_0.3.0
[52] magrittr_1.5        scales_1.0.0        stringi_1.4.3
[55] latticeExtra_0.6-28 sandwich_2.5-1      Formula_1.2-3
[58] TH.data_1.0-10      lava_1.6.5          RColorBrewer_1.1-2
[61] iterators_1.0.10    tools_3.6.0         cmprsk_2.2-8
[64] glue_1.3.1          purrr_0.3.2         survival_2.44-1.1
[67] colorspace_1.4-1    cluster_2.0.9       knitr_1.23
[70] quantreg_5.40
\end{verbatim}


Setup data
\lstset{language=r,label= ,caption= ,captionpos=b,numbers=none}
\begin{lstlisting}
set.seed(18)
astrain <- simActiveSurveillance(278)
astest <- simActiveSurveillance(208)
astrain[,Y1:=1*(event==1 & time<=1)]
astest[,Y1:=1*(event==1 & time<=1)]
lrfit.ex <- glm(Y1~age+lpsaden+ppb5+lmax+ct1+diaggs,data=astrain,family="binomial")
lrfit.inc <- glm(Y1~age+lpsaden+ppb5+lmax+ct1+diaggs+erg.status,data=astrain,family="binomial")
## Score(list("Exclusive ERG"=lrfit.ex,"Inclusive ERG"=lrfit.inc),data=astest,formula=Y1~1,se.fit=0L,metrics="brier",contrasts=FALSE)
\end{lstlisting}

Now also works with for \texttt{bootcv} without errors, now also returning no-NA IPA. These are negativ, however, which I don't know if make sense.
\lstset{language=r,label= ,caption= ,captionpos=b,numbers=none}
\begin{lstlisting}
X1 <- Score(list("Exclusive ERG"=lrfit.ex,"Inclusive ERG"=lrfit.inc),data=astest,
	    formula=Y1~1,summary="ipa",se.fit=0L,metrics="brier",contrasts=FALSE,
	    split.method = "bootcv", B=100)
\end{lstlisting}


\lstset{language=r,label= ,caption= ,captionpos=b,numbers=none}
\begin{lstlisting}
X1
\end{lstlisting}

\begin{verbatim}

Metric Brier:

Results by model:

           model Brier     IPA
1:    Null model 0.157  0.0000
2: Exclusive ERG 0.169 -0.0781
3: Inclusive ERG 0.163 -0.0396

Bootstrap cross-validation based on 100 bootstrap samples (drawn with replacement) each of size 208.
\end{verbatim}


And gives some result for \texttt{cv} when just using the same method as for \texttt{bootcv}. Not sure these are correct however.
\lstset{language=r,label= ,caption= ,captionpos=b,numbers=none}
\begin{lstlisting}
X1 <- Score(list("Exclusive ERG"=lrfit.ex,"Inclusive ERG"=lrfit.inc),data=astest,
	    formula=Y1~1,summary="ipa",se.fit=0L,metrics="brier",contrasts=FALSE,
	    split.method = "cv5", B=100)
\end{lstlisting}


\lstset{language=r,label= ,caption= ,captionpos=b,numbers=none}
\begin{lstlisting}
X1
\end{lstlisting}

\begin{verbatim}

Metric Brier:

Results by model:

           model Brier      IPA
1:    Null model 0.154  0.00000
2: Exclusive ERG 0.160 -0.03601
3: Inclusive ERG 0.153  0.00392

5-fold cross-validation repeated 100 times.
\end{verbatim}
\end{document}